% Add document class and packages
\documentclass{article}
\usepackage[utf8]{inputenc}
\usepackage{geometry}
\usepackage{hyperref}
\usepackage{amsmath,amssymb}
\usepackage{graphicx}
\usepackage{booktabs}
\usepackage{ifthen}

% ---------------------------------------------------------------------------
% Title
% ---------------------------------------------------------------------------
\title{Secret Raffle: A Privacy-Preserving Lottery dApp on Secret Network}
\author{Henry Borthwick \\ COSC 473 Term Project 2025}
\date{\today}

\begin{document}
\maketitle

% ---------------------------------------------------------------------------
% Abstract
% ---------------------------------------------------------------------------
\begin{abstract}
% Public blockchain raffles typically expose every ticket purchase and rely on off-chain scripts to choose winners, eroding both participant privacy and trust in the outcome.  We present \emph{Secret Raffle}, a fully on-chain, end-to-end verifiable lottery built with CosmWasm and deployed on the Secret Network.  Our smart-contract architecture segregates privileges (admin \emph{vs.} player) and leverages Secret Network's encrypted state to hide individual ticket balances while escrow-ing the pot in native \texttt{uSCRT}.  After a configurable end-time the contract derives pseudo-randomness from the encrypted block seed to deterministically select a winner proportional to ticket holdings.  The winner can atomically claim the prize and, using a viewing-key permit, privately retrieve a hidden \emph{secret phrase}—a mechanic that can serve as a redemption code or proof-of-win in downstream experiences.

% A React/TypeScript front-end built with Vite and SecretJS abstracts wallet connection, ticket purchase, and results display into three intuitive clicks.  End-to-end tests executed on pulsar-3 testnet demonstrate a \textless200 kGas ticket purchase, \textless300 kGas winner selection, and successful prize payouts for up to \(10^{6}\) tickets without overflow.  These results validate that privacy-preserving raffles can be delivered with minimal contract complexity while retaining familiar user-experience patterns.  The project serves as a blueprint for more sophisticated, fair-chance games and fund-raising mechanisms in the Cosmos ecosystem.
\end{abstract}

% ---------------------------------------------------------------------------
% Problem Statement & Motivation 1/2 page
% ---------------------------------------------------------------------------
\section{Problem Statement \& Motivation}

\subsection{Problem Statement}

\subsubsection{Centralised Raffles: A Broken Trust Model}
Traditional online raffles---whether used for charity fund-raising, community giveaways, or customer engagement campaigns---are almost always operated by a central authority who (1) collects entry fees, (2) draws a winner off-chain, and (3) promises to distribute the prize. Participants must blindly trust that the organiser will not tamper with the draw, exclude entries, or walk away with the pot. Even projects that migrate ticket sales on-chain (e.g., Ethereum ERC-20 based lotteries) still expose a major weakness: every purchase is publicly traceable, so whales can copy trade large entries, and winner selection typically relies on external random-number services or opaque scripts.

\subsubsection{The Need for Verifiable \emph{and} Private Lotteries}
Communities and creators increasingly want ways to raise funds or distribute rewards in a manner that is both provably fair \emph{and} respectful of participants' privacy. A transparent contract is essential for auditability, yet publishing every wallet's ticket balance can deter users who prefer to keep spending habits or donation sizes confidential. Likewise, sourcing randomness must not introduce new trust assumptions or central points of failure.

\subsection{Motivation}

\subsubsection{Why Secret Network + CosmWasm?}
Secret Network brings encrypted contract state and compute to the Cosmos ecosystem. By storing ticket counts and the raffle's secret phrase inside the Trusted Execution Environment (TEE), our \emph{Secret Raffle} contract simultaneously delivers:
\begin{itemize}
  \item \textbf{Privacy:} Individual ticket balances are hidden from everyone except the rightful owner, revealed only through permissioned queries (\texttt{Permit}) after wallet signature.
  \item \textbf{Trustless Funds Escrow:} All \texttt{uSCRT} payments are held by the contract; payout executes atomically when the winner calls \texttt{claim\_prize}. No organiser custody is ever required.
  \item \textbf{On-chain Randomness:} The contract derives a pseudo-random seed from Secret Network's encrypted block entropy, eliminating reliance on third-party VRF services.
  \item \textbf{Role-based Access Control:} Only the admin can configure and start the raffle, while any user may purchase tickets; only the selected winner can claim the pot.
\end{itemize}
These properties address both fairness and privacy gaps in existing lottery implementations, providing a robust fund-raising primitive that communities can trust without sacrificing anonymity.

% ---------------------------------------------------------------------------
% Related Work & Context 1/2 page
% ---------------------------------------------------------------------------
\section{Related Work \& Context}

\subsection{Context}
Blockchain lotteries aim to offer fair and transparent prize distribution, but existing solutions compromise either privacy or trust. Public ledgers expose user activity, while privacy-focused designs often rely on off-chain randomness or trusted parties, reducing verifiability. \emph{Secret Raffle} addresses these gaps by combining:
\begin{itemize}
  \item \emph{Encrypted state} for confidential ticket balances.
  \item \emph{Enclave-derived randomness} to eliminate oracle dependence.
  \item \emph{Self-contained winner selection and payout} within a single CosmWasm contract.
  \item \emph{Standardised permit queries} for private winner verification.
\end{itemize}
This design offers a unique blend of privacy and verifiability, absent in current Cosmos ecosystem lotteries.

% Surveying existing solutions
\subsection{Related Work}

\subsubsection{Public On-Chain Lotteries}
\begin{itemize}
  \item \textbf{PoolTogether (Ethereum):} A ``no-loss'' lottery pooling deposits into interest-bearing protocols, with winners selected via Chainlink VRF. It ensures transparency but exposes transactions publicly and depends on a trusted oracle~\cite{PoolTogether}.
  \item \textbf{PancakeSwap Lottery (BNB Smart Chain):} Users buy tickets with CAKE tokens, and winners are chosen using block hash entropy. Ticket counts are visible, enabling behavior analysis and strategic exploitation~\cite{PancakeSwap}.
  \item \textbf{JunoLottery (Cosmos/Juno):} An open-source CosmWasm raffle contract, limited by public-state privacy issues and reliance on external winner selection~\cite{JunoLottery}.
\end{itemize}

\subsubsection{Privacy-Focused Blockchain Games}
\begin{itemize}
  \item \textbf{SecretScratchCard (Secret Network):} Showcases private NFT ownership but uses off-chain randomness and lacks communal fund management~\cite{SecretScratchCard}.
  \item \textbf{Shade Protocol's ``Silk Raffle'':} Awarded tickets to SILK stakers, with winners selected via a trusted script, limiting transparency~\cite{ShadeRaffle}.
\end{itemize}

% ---------------------------------------------------------------------------
% Defining the Architecture & Contract Design section
% ---------------------------------------------------------------------------
\section{Architecture \& Contract Design}

\subsection{Architecture}
Figure \ref{fig:Frontend Contract Interaction Architecture Structure Diagram Appendix} shows the frontend contract interaction architecture structure diagram for the dApp.

\subsection{Contract Design}

\subsubsection{ExecuteMsg}
\begin{tabular}{@{}lp{9cm}@{}}
  \textbf{set\_raffle} & Admin-only. Sets \texttt{ticket\_price}, \texttt{end\_time}, and \texttt{secret} phrase. \\
  \textbf{start\_raffle} & Opens raffle for ticket purchases. \\
  \textbf{buy\_ticket} & Public. Accepts \texttt{uSCRT}; mints tickets per \texttt{ticket\_price}. \\
  \textbf{select\_winner} & Callable post-\texttt{end\_time}. Picks winner based on ticket holdings. \\
  \textbf{claim\_prize} & Winner claims pot and unlocks secret viewing. \\
\end{tabular}

\subsubsection{QueryMsg}
\begin{center}
\begin{tabular}{@{}llll@{}}
\toprule
Variant & Fields & Access & Purpose \\
\midrule
\texttt{raffle\_info} & -- & Public & Raffle summary \\
\texttt{with\_permit} $\rightarrow$ \texttt{get\_secret} & \texttt{Permit} & Winner & Reveal secret phrase \\
\texttt{with\_permit} $\rightarrow$ \texttt{get\_tickets} & \texttt{Permit} & Ticket holder & View own ticket count \\
\bottomrule
\end{tabular}
\end{center}

\subsubsection{State Layout}
\begin{center}
\begin{tabular}{@{}lll@{}}
\toprule
Key & Type & Purpose \\
\midrule
\texttt{ADMIN} & \texttt{Item<CanonicalAddr>} & Contract admin \\
\texttt{RAFFLE} & \texttt{Item<Raffle>} & Config: end time, price, secret \\
\texttt{TICKETS} & \texttt{Keymap<CanonicalAddr,u64>} & User ticket balances \\
\texttt{TOTAL\_TICKETS} & \texttt{Item<u64>} & Total tickets sold \\
\texttt{WINNER} & \texttt{Item<Option<CanonicalAddr>>} & Selected winner \\
\texttt{RAFFLE\_STARTED} & \texttt{Item<bool>} & Raffle status \\
\texttt{WINNER\_SELECTED} & \texttt{Item<bool>} & Winner selection status \\
\texttt{PRIZE\_CLAIMED} & \texttt{Item<bool>} & Prize claim status \\
\bottomrule
\end{tabular}
\end{center}

\subsubsection{Special Features}
The contract leverages Secret Network's privacy stack and demonstrates several advanced patterns:
\begin{itemize}
  \item \textbf{Encrypted State:} Ticket counts and the hidden secret phrase are stored inside the enclave, never appearing on the public ledger.
  \item \textbf{Permit-Based Queries:} Access to private data requires a wallet-signed \texttt{Permit}, ensuring only authorised users can view their tickets or the secret.
  \item \textbf{On-chain Randomness (Stretch Goal):} Winner selection draws entropy from the encrypted block seed, showcasing a fully on-chain random draw without external VRF services.
  \item \textbf{Timelock Pattern (Stretch Goal):} Privileged actions such as \texttt{select\_winner} are gated by an \texttt{end\_time} timestamp, illustrating time-delayed, opt-out execution flows common in DAO governance.
  \item \textbf{Modular Monorepo Architecture:} The repository bundles contract, CLI uploader, and React UI as separate workspaces. This layout is ready for multi-contract expansion (e.g., CW20 ticket token or marketplace) while keeping dev-ops simple.
  \item \textbf{IBC / Cross-chain Ready:} Built on \texttt{secret\_cosmwasm\_std~1.1}, the contract can be extended with IBC message handlers to escrow prizes across Cosmos zones.
  \item \textbf{Developer UX Extras:} Auto-generated JSON schemas, typed SecretJS hooks, and detailed architecture diagrams accelerate contributor onboarding and auditing.
\end{itemize}

\subsubsection{Security Considerations}
\begin{center}
\begin{tabular}{@{}lp{10cm}@{}}
\toprule
Aspect & Design Notes \\
\midrule
Access Control & Admin-only actions restricted; UI conceals options for non-admins. \\
Edge Cases & Zero-ticket scenarios handled without errors. \\
Randomness & Block entropy used for demo; upgradable to Secret Randomness API. \\
Privacy & Encrypted state with permit-only access. \\
Upgradability & State design supports future enhancements without breaking changes. \\
\bottomrule
\end{tabular}
\end{center}

% ---------------------------------------------------------------------------
% Frontend Workflow & UX 1/4 page
% ---------------------------------------------------------------------------
\section{Frontend Workflow \& UX}
The user and admin frontend interaction workflow and user experience is represented using a flow diagrami n Figure \ref{fig:Frontend Workflow Appendix}.

% ---------------------------------------------------------------------------
% Implementation Choices & Tradeoffs 1/4 page
% ---------------------------------------------------------------------------
\section{Implementation Choices \& Tradeoffs}

\begin{itemize}
  \item \textbf{Flag-based Lifecycle Tracking}\,: Instead of a single enum that serialises the raffle's current stage, the contract stores three independent boolean flags (\texttt{RAFFLE\_STARTED}, \texttt{WINNER\_SELECTED}, \texttt{PRIZE\_CLAIMED}).  This micro–state-machine requires one extra storage slot, yet it simplifies future upgrades because adding a new phase is a non-breaking change—no migration logic is needed to reinterpret an amended enum.  The \(<40~B\) extra storage is negligible compared with Secret Network's 16\,k gas read cost, so we prioritised forward-compatibility over byte-efficiency.

  \item \textbf{Canonically-Keyed \texttt{TICKETS} Map}\,: Ticket balances are stored in a key-value map from \texttt{CanonicalAddr} \("compressed" 20-byte addresses\) to \texttt{u64}.  Canonical keys are deterministic and shorter than human strings, saving gas on every insert.  The alternative—storing a \texttt{Vec<(Addr,u64)>}—would allow constant-time iteration for winner selection but incur an \(O(n)\) scan on \\emph{every} ticket purchase.  Our design flips the cost profile:  writes are \(O(1)\) and the \(O(n)\) scan only happens once at raffle end, which is acceptable given a hard cap of \(10^{6}\) tickets in tests.

  \item \textbf{On-chain Randomness via \texttt{env.block.random}}\,: The winner is picked by hashing the encrypted block seed and modding by \texttt{TOTAL\_TICKETS}.  This removes any dependency on external VRF oracles (lower latency, no additional fees), at the cost of a small modulo bias and the fact that a block proposer with \(<1\,\%\) of total stake could, in theory, try to grind an advantageous seed.  Because the maximum prize in our demo is intentionally small (\(<10~SCRT)\) and the contract is educational, we judged the attack surface acceptable.  A future upgrade path is documented to integrate Secret Network's dedicated Randomness API for higher guarantees.

  \item \textbf{Permit-Based Private Queries}\,: All sensitive reads (\texttt{get\_tickets}, \texttt{get\_secret}) require a SNIP-24 \emph{permit}—a stateless, single-signature authorisation included in the query payload.  This keeps balances private while avoiding the UX friction of persistent viewing keys.  The downside is a larger query payload and one extra Keplr pop-up, but during user testing the added click was preferred over managing secrets manually.

  % Optional — comment out if space needed.
  % \item \textbf{Monorepo Workspaces}\,: Contract, CLI uploader, and React UI live in a single repo with Cargo/NPM workspaces.  This streamlines type-sharing and CI yet creates larger dependency graphs; a mono-repo is acceptable for a small-scale university project but would be split in a production setting.
\end{itemize}

% ---------------------------------------------------------------------------
% Future Work & Roadmap 1/4 page
% ---------------------------------------------------------------------------
\section{Future Work \& Roadmap}

\paragraph{v1.1 – Security \& Reliability}
\begin{itemize}
  \item \textbf{Secret Randomness API}: Replace the demo modulo selection with Secret Network's VRF-style Randomness API to remove modulo bias and proposer grinding risk.
  \item \textbf{Comprehensive Test-Suite}: Add property-based fuzz tests (\texttt{cw\_vm\_test}) and differential integration checks against a mock reference implementation.
  \item \textbf{Formal Audit}: Engage Scrt Labs' community auditors; action high-severity findings before main-net launch.
\end{itemize}

\paragraph{v1.2 – Tokenisation \& Marketplace}
\begin{itemize}
  \item \textbf{SNIP-20 "RAFFLE" Ticket Token}: Mint tradable tickets, enabling secondary-market transfers and on-chain proof of ownership.
  \item \textbf{Multi-Contract Split}: Move ticket logic into its own contract; the raffle contract reads balances via cross-contract queries—improves separation of concerns and gas isolation.
  \item \textbf{Fixed-Price Marketplace}: Simple order-book contract so players can list tickets pre-draw; provides price discovery and liquidity.
\end{itemize}

\paragraph{v2.0 – Cross-Chain \& Governance}
\begin{itemize}
  \item \textbf{IBC Escrow}: Allow deposits in any ICS-20 token (ATOM, NTRN) with prizes returned over IBC; widens target user base beyond Secret Network.
  \item \textbf{DAO-Controlled Raffle Factory}: A factory contract that mints new raffle instances via on-chain proposals, letting communities spin up private lotteries without code changes.
  \item \textbf{Upgradeable Contracts}: Adopt CosmWasm's migration pattern plus an on-chain timelock so future versions can be rolled out trustlessly.
\end{itemize}

\paragraph{Stretch Goals}
\begin{itemize}
  \item NFT prizes and sponsorship integrations (SNIP-721).
  \item Verifiable front-end bundle hash pinned to IPFS/Skynet for full end-to-end integrity.
  \item Mobile-first PWA with push notifications when a raffle is drawn.
\end{itemize}

% ---------------------------------------------------------------------------
% References & Citations
% ---------------------------------------------------------------------------
\section{References \& Citations}
\begin{thebibliography}{9}

\bibitem{PoolTogether}
PoolTogether. ``A no-loss lottery built on Ethereum.'' Available: \url{https://www.pooltogether.com/}

\bibitem{PancakeSwap}
PancakeSwap. ``Lottery.'' Documentation. Available: \url{https://docs.pancakeswap.finance/products/lottery}

\bibitem{JunoLottery}
``JunoLottery.'' GitHub repository. Available: \url{https://github.com/CosmosContracts/juno-lottery}

\bibitem{SecretScratchCard}
``SecretScratchCard.'' GitHub repository. Available: \url{https://github.com/scrtlabs/SecretScratchcard}

\bibitem{ShadeRaffle}
Shade Protocol. ``Silk Raffle.'' Project announcements, 2023.

\end{thebibliography}

\newpage

% ---------------------------------------------------------------------------
% Appendix
% ---------------------------------------------------------------------------
\section{Appendix}

\subsection{Frontend Contract Interaction Architecture Structure Diagram}
\begin{figure}[h]
  \vspace{-0.5cm}
  \hspace{-2cm}
  \includegraphics[width=1.3\textwidth]{Secret Raffle Monorepo Frontend Contract Interaction Diagram.png}
  \caption{Frontend Contract Interaction Architecture Structure Diagram.}
  \label{fig:Frontend Contract Interaction Architecture Structure Diagram Appendix}
\end{figure}

\newpage

\subsection{Frontend Workflow Diagram}
\begin{figure}[h]
  \vspace{0cm}
  \hspace{0cm}
  \includegraphics[width=0.6\textwidth]{frontend-workflow.png}
  \caption{Frontend Workflow.}
  \label{fig:Frontend Workflow Appendix}
\end{figure}

\newpage

\subsection{dApp Contract, Uploader, and Frontend Interaction Architecture Structure Diagram}
\begin{figure}[h]
  \vspace{-0.5cm}
  \hspace{-3cm}
  \includegraphics[width=1.4\textwidth]{Secret Raffle Monorepo Structure Diagram.png}
  \caption{dApp Contract, Uploader, and Frontend Architecture Interaction Structure Diagram.}
  \label{fig:dApp Contract, Uploader, and Frontend Architecture Interaction Structure Diagram Appendix}
\end{figure}

\subsection{dApp Contract, Uploader, and Frontend Architecture Structure Diagram}
\begin{figure}[h]
  \vspace{0cm}
  \hspace{-3cm}
  \includegraphics[width=1.4\textwidth]{Secret Raffle Monorepo Structure Diagram v2.png}
  \caption{dApp Contract, Uploader, and Frontend Architecture Structure Diagram.}
  \label{fig:dApp Contract, Uploader, and Frontend Architecture Structure Diagram Appendix}
\end{figure}

\newpage

\subsection{Contract Architecture Structure Diagram}
\begin{figure}[h]
  \vspace{0cm}
  \hspace{-3cm}
  \includegraphics[width=1.4\textwidth]{Secret Raffle Monorepo Contract Structure Diagram.png}
  \caption{Contract Architecture Structure Diagram.}
  \label{fig:Contract Architecture Structure Diagram Appendix}
\end{figure}

\subsection{Uploader Architecture Structure Diagram}
\begin{figure}[h]
  \vspace{0cm}
  \hspace{-3cm}
  \includegraphics[width=1.4\textwidth]{Secret Raffle Monorepo Uploader Structure Diagram.png}
  \caption{Uploader Architecture Structure Diagram.}
  \label{fig:Uploader Architecture Structure Diagram Appendix}
\end{figure}

\newpage

\subsection{Frontend Architecture Structure Diagram}
\begin{figure}[h]
  \vspace{0cm}
  \hspace{-3cm}
  \includegraphics[width=1.4\textwidth]{Secret Raffle Monorepo Frontend Structure Diagram.png}
  \caption{Frontend Architecture Structure Diagram.}
  \label{fig:Frontend Architecture Structure Diagram Appendix}
\end{figure}

\end{document}
