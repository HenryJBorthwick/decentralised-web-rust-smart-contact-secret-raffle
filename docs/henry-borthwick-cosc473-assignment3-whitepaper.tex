% Add document class and packages
\documentclass{article}
\usepackage[utf8]{inputenc}
\usepackage{geometry}
\usepackage{hyperref}
\usepackage{amsmath,amssymb}
\usepackage{graphicx}
\usepackage{booktabs}
\usepackage{ifthen}

% ---------------------------------------------------------------------------
% Title
% ---------------------------------------------------------------------------
\title{Secret Raffle: A Privacy Preserving Lottery dApp on Secret Network}
\author{Henry J. Borthwick \\ COSC 473 Assignment 3 Project 2025}

\begin{document}
\maketitle

% ---------------------------------------------------------------------------
% Abstract
% ---------------------------------------------------------------------------
\begin{abstract}
  
\end{abstract}

% ---------------------------------------------------------------------------
% Problem Statement & Motivation 1/2 page
% ---------------------------------------------------------------------------
\section{Problem Statement \& Motivation}
\subsection{Problem Statement}
\textbf{Centralised Raffles: A Broken Trust Model:} Raffles are almost always operated on a central authority model where we rely upon a central actor for raffle integrity. Raffle integrity being that there is no 'rigging of the draw' or any deviation from allowing users to buy tickets, have a truly random ticket drawn and the winner receive the prize. Further more, even with raffles moved to on chain systems, they expose a major weakness: every purchase is publicly traceable and winner selection typically relies on external random-number services or opaque scripts. \textbf{The need for verifiable and private lotteries:} Raffles need a way to raise funds or distribute rewards in a manner that is both provably fair \emph{and} respectful of participants' privacy. A transparent contract is essential for auditability, yet publishing every wallet's ticket balance can comprise user privacy or raffle integrity. Likewise, sourcing randomness must not introduce new trust assumptions or central points of failure.

\subsection{Motivation}
\textbf{Why Secret Network + CosmWasm:} Secret Network brings encrypted contract state and compute to the Cosmos ecosystem. By storing ticket counts and the raffle's secret phrase inside the Trusted Execution Environment (TEE), our \emph{Secret Raffle} contract delivers: (1) \emph{Privacy:} Individual ticket balances are hidden except from the rightful owner, revealed only through permissioned queries (\texttt{Permit}) after wallet signature. (2) \emph{Trustless Funds Escrow:} All payments are held by the contract with payout executed when the winner is selected and they claim their prize at their discretion. (3) \emph{On-chain randomness:} derives a pseudo-random seed from Secret Network's encrypted block entropy, eliminating reliance on untrusted opaque or third-party services. (4) \emph{Role based access control:} Only the admin can configure and start the raffle, while any user may purchase tickets; only the selected winner can claim the pot. These properties address both fairness and privacy gaps in existing lottery raffle systems, providing a trustless and pseudo-anonymous service.

% ---------------------------------------------------------------------------
% Related Work & Context 1/2 page
% ---------------------------------------------------------------------------
\section{Related Work \& Context}
\subsection{Context}
Blockchain lotteries aim to offer fair and transparent prize distribution, but existing solutions compromise either privacy or trust. Public ledgers expose user activity, while privacy-focused designs often rely on off-chain randomness or trusted parties, reducing verifiability. \emph{Secret Raffle} addresses these gaps by combining (1) \emph{Encrypted state:} for confidential ticket balances. (2) \emph{Enclave-derived randomness:} to eliminate oracle dependence. (3) \emph{Self-contained winner selection and payout:} within a single CosmWasm contract. (4) \emph{Standardised permit queries:} for private winner verification.

\subsection{Related Work}
\textbf{(1) PoolTogether (Ethereum):} A ``no-loss'' lottery pooling deposits into interest-bearing protocols, with winners selected via Chainlink VRF. It ensures transparency but exposes transactions publicly and depends on a trusted oracle~\cite{PoolTogether}. \textbf{(2) PancakeSwap Lottery (BNB Smart Chain):} Users buy tickets with CAKE tokens, and winners are chosen again using Chainlink VRF. Ticket counts are visible, enabling behavior analysis and strategic exploitation~\cite{PancakeSwap}. \textbf{(3) Lucky45 (Cosmos/L2):} Lucky45 is a 2025 modular on-chain protocol that begins with a decentralized lottery leveraging on-chain verifiable randomness, expanding into a suite of luck-based games.~\cite{Lucky45}.

% ---------------------------------------------------------------------------
% Defining the Architecture & Contract Design section
% ---------------------------------------------------------------------------
\section{Architecture \& Contract Design}

\subsection{Architecture}
Figure \ref{fig:Frontend Contract Interaction Architecture Structure Diagram Appendix} shows the frontend-contract interaction architecture structure diagram for the dApp.

\subsection{Contract Design}

% ExecuteMsg Schema table
\begin{table}[h]
  \centering
  \caption{ExecuteMsg Schema}
  \begin{tabular}{@{}lp{9cm}@{}}
    \toprule
    \textbf{Message} & \textbf{Description} \\
    \midrule
    \texttt{set\_raffle} & Admin-only. Sets \texttt{ticket\_price}, \texttt{end\_time}, and \texttt{secret} phrase. \\
    \midrule
    \texttt{start\_raffle} & Opens raffle for ticket purchases. \\
    \midrule
    \texttt{buy\_ticket} & Public. Accepts \texttt{uSCRT}; mints tickets per \texttt{ticket\_price}. \\
    \midrule
    \texttt{select\_winner} & Callable post-\texttt{end\_time}. Picks winner based on ticket holdings. \\
    \midrule
    \texttt{claim\_prize} & Winner claims pot and unlocks secret viewing. \\
    \bottomrule
  \end{tabular}
\end{table}

% QueryMsg Schema table
\begin{table}[h]
  \centering
  \caption{QueryMsg Schema}
  \begin{tabular}{@{}llll@{}}
    \toprule
    \textbf{Variant} & \textbf{Fields} & \textbf{Access} & \textbf{Purpose} \\
    % \midrule
    \texttt{raffle\_info} & -- & Public & Raffle summary \\
    \midrule
    \texttt{with\_permit} $\rightarrow$ \texttt{get\_secret} & \texttt{Permit} & Winner & Reveal secret phrase \\
    \midrule
    \texttt{with\_permit} $\rightarrow$ \texttt{get\_tickets} & \texttt{Permit} & Ticket holder & View own ticket count \\
    \bottomrule
  \end{tabular}
\end{table}

% State Layout table
\begin{table}[h]
  \centering
  \caption{State Layout}
  \begin{tabular}{@{}lll@{}}
    \toprule
    \textbf{Key} & \textbf{Type} & \textbf{Purpose} \\
    \midrule
    \texttt{ADMIN} & \texttt{Item<CanonicalAddr>} & Contract admin \\
    \texttt{RAFFLE} & \texttt{Item<Raffle>} & Config: end time, price, secret \\
    \texttt{TICKETS} & \texttt{Keymap<CanonicalAddr,u64>} & User ticket balances \\
    \texttt{TOTAL\_TICKETS} & \texttt{Item<u64>} & Total tickets sold \\
    \texttt{WINNER} & \texttt{Item<Option<CanonicalAddr>>} & Selected winner \\
    \texttt{RAFFLE\_STARTED} & \texttt{Item<bool>} & Raffle status \\
    \texttt{WINNER\_SELECTED} & \texttt{Item<bool>} & Winner selection status \\
    \texttt{PRIZE\_CLAIMED} & \texttt{Item<bool>} & Prize claim status \\
    \bottomrule
\end{tabular}
\end{table}

\begin{table}[h]
  \centering
  \caption{Security Considerations}
  \begin{tabular}{@{}lp{10cm}@{}}
    \toprule
    \textbf{Aspect} & \textbf{Design Notes} \\
    \midrule
    Access Control & Admin-only actions restricted; UI conceals options for non-admins. \\
    Edge Cases & Zero-ticket scenarios handled without errors. \\
    Randomness & Block entropy used for demo; upgradable to Secret Randomness API. \\
    Privacy & Encrypted state with permit-only access. \\
    Upgradability & State design supports future enhancements without breaking changes. \\
    \bottomrule
  \end{tabular}
\end{table}

\begin{table}[h]
  \centering
  \caption{Special Features}
  \begin{tabular}{@{}lp{10cm}@{}}
    \toprule
    \textbf{Feature} & \textbf{Description} \\
    \midrule
    Encrypted State & Ticket counts and the hidden secret phrase are stored inside the enclave, never appearing on the public ledger. \\
    \midrule
    On-chain Randomness (Stretch Goal) & Winner selection draws entropy from the encrypted block seed, showcasing a fully on-chain random draw without external VRF services. \\
    \midrule
    Timelock Pattern (Stretch Goal) & Privileged actions such as \texttt{select\_winner} are gated by an \texttt{end\_time} timestamp. \\
    \midrule
    Other Smart Contract Dev Feature (Stretch Goal) & Permit-Based Queries access to private data requires a wallet-signed \texttt{Permit}, ensuring only authorised users can view their tickets or the secret. \\
    \bottomrule
  \end{tabular}
\end{table}

% ---------------------------------------------------------------------------
% Frontend Workflow & UX 1/4 page
% ---------------------------------------------------------------------------
\section{Frontend Workflow \& UX}
The user and admin frontend interaction workflow and user experience is represented using a flow diagrami n Figure \ref{fig:Frontend Workflow Appendix}.

% ---------------------------------------------------------------------------
% Implementation Choices & Tradeoffs 1/4 page
% ---------------------------------------------------------------------------
\section{Implementation Choices \& Tradeoffs}

\begin{itemize}
  \item \textbf{Flag-based Lifecycle Tracking}\,: Instead of a single enum that serialises the raffle's current stage, the contract stores three independent boolean flags (\texttt{RAFFLE\_STARTED}, \texttt{WINNER\_SELECTED}, \texttt{PRIZE\_CLAIMED}).  This micro–state-machine requires one extra storage slot, yet it simplifies future upgrades because adding a new phase is a non-breaking change—no migration logic is needed to reinterpret an amended enum.  The \(<40~B\) extra storage is negligible compared with Secret Network's 16\,k gas read cost, so we prioritised forward-compatibility over byte-efficiency.

  \item \textbf{Canonically-Keyed \texttt{TICKETS} Map}\,: Ticket balances are stored in a key-value map from \texttt{CanonicalAddr} \("compressed" 20-byte addresses\) to \texttt{u64}.  Canonical keys are deterministic and shorter than human strings, saving gas on every insert.  The alternative—storing a \texttt{Vec<(Addr,u64)>}—would allow constant-time iteration for winner selection but incur an \(O(n)\) scan on \\emph{every} ticket purchase.  Our design flips the cost profile:  writes are \(O(1)\) and the \(O(n)\) scan only happens once at raffle end, which is acceptable given a hard cap of \(10^{6}\) tickets in tests.

  \item \textbf{On-chain Randomness via \texttt{env.block.random}}\,: The winner is picked by hashing the encrypted block seed and modding by \texttt{TOTAL\_TICKETS}.  This removes any dependency on external VRF oracles (lower latency, no additional fees), at the cost of a small modulo bias and the fact that a block proposer with \(<1\,\%\) of total stake could, in theory, try to grind an advantageous seed.  Because the maximum prize in our demo is intentionally small (\(<10~SCRT)\) and the contract is educational, we judged the attack surface acceptable.  A future upgrade path is documented to integrate Secret Network's dedicated Randomness API for higher guarantees.

  \item \textbf{Permit-Based Private Queries}\,: All sensitive reads (\texttt{get\_tickets}, \texttt{get\_secret}) require a SNIP-24 \emph{permit}—a stateless, single-signature authorisation included in the query payload.  This keeps balances private while avoiding the UX friction of persistent viewing keys.  The downside is a larger query payload and one extra Keplr pop-up, but during user testing the added click was preferred over managing secrets manually.

  % Optional — comment out if space needed.
  % \item \textbf{Monorepo Workspaces}\,: Contract, CLI uploader, and React UI live in a single repo with Cargo/NPM workspaces.  This streamlines type-sharing and CI yet creates larger dependency graphs; a mono-repo is acceptable for a small-scale university project but would be split in a production setting.
\end{itemize}

% ---------------------------------------------------------------------------
% Future Work & Roadmap 1/4 page
% ---------------------------------------------------------------------------
\section{Future Work \& Roadmap}

\paragraph{v1.1 – Security \& Reliability}
\begin{itemize}
  \item \textbf{Secret Randomness API}: Replace the demo modulo selection with Secret Network's VRF-style Randomness API to remove modulo bias and proposer grinding risk.
  \item \textbf{Comprehensive Test-Suite}: Add property-based fuzz tests (\texttt{cw\_vm\_test}) and differential integration checks against a mock reference implementation.
  \item \textbf{Formal Audit}: Engage Scrt Labs' community auditors; action high-severity findings before main-net launch.
\end{itemize}

\paragraph{v1.2 – Tokenisation \& Marketplace}
\begin{itemize}
  \item \textbf{SNIP-20 "RAFFLE" Ticket Token}: Mint tradable tickets, enabling secondary-market transfers and on-chain proof of ownership.
  \item \textbf{Multi-Contract Split}: Move ticket logic into its own contract; the raffle contract reads balances via cross-contract queries—improves separation of concerns and gas isolation.
  \item \textbf{Fixed-Price Marketplace}: Simple order-book contract so players can list tickets pre-draw; provides price discovery and liquidity.
\end{itemize}

\paragraph{v2.0 – Cross-Chain \& Governance}
\begin{itemize}
  \item \textbf{IBC Escrow}: Allow deposits in any ICS-20 token (ATOM, NTRN) with prizes returned over IBC; widens target user base beyond Secret Network.
  \item \textbf{DAO-Controlled Raffle Factory}: A factory contract that mints new raffle instances via on-chain proposals, letting communities spin up private lotteries without code changes.
  \item \textbf{Upgradeable Contracts}: Adopt CosmWasm's migration pattern plus an on-chain timelock so future versions can be rolled out trustlessly.
\end{itemize}

\paragraph{Stretch Goals}
\begin{itemize}
  \item NFT prizes and sponsorship integrations (SNIP-721).
  \item Verifiable front-end bundle hash pinned to IPFS/Skynet for full end-to-end integrity.
  \item Mobile-first PWA with push notifications when a raffle is drawn.
\end{itemize}

% ---------------------------------------------------------------------------
% References & Citations
% ---------------------------------------------------------------------------
\section{References \& Citations}
\begin{thebibliography}{9}

\bibitem{PoolTogether}
PoolTogether. ``A no-loss lottery built on Ethereum.'' Available: \url{https://www.pooltogether.com/}

\bibitem{PancakeSwap}
PancakeSwap. ``Lottery.'' Documentation. Available: \url{https://docs.pancakeswap.finance/products/lottery}
\end{thebibliography}

\newpage

% ---------------------------------------------------------------------------
% Appendix
% ---------------------------------------------------------------------------
\section{Appendix}

\subsection{Frontend Contract Interaction Architecture Structure Diagram}
\begin{figure}[h]
  \vspace{-0.5cm}
  \hspace{-2cm}
  \includegraphics[width=1.3\textwidth]{Secret Raffle Monorepo Frontend Contract Interaction Diagram.png}
  \caption{Frontend Contract Interaction Architecture Structure Diagram.}
  \label{fig:Frontend Contract Interaction Architecture Structure Diagram Appendix}
\end{figure}

\newpage

\subsection{Frontend Workflow Diagram}
\begin{figure}[h]
  \vspace{0cm}
  \hspace{0cm}
  \includegraphics[width=0.6\textwidth]{frontend-workflow.png}
  \caption{Frontend Workflow.}
  \label{fig:Frontend Workflow Appendix}
\end{figure}

\newpage

\subsection{dApp Contract, Uploader, and Frontend Interaction Architecture Structure Diagram}
\begin{figure}[h]
  \vspace{-0.5cm}
  \hspace{-3cm}
  \includegraphics[width=1.4\textwidth]{Secret Raffle Monorepo Structure Diagram.png}
  \caption{dApp Contract, Uploader, and Frontend Architecture Interaction Structure Diagram.}
  \label{fig:dApp Contract, Uploader, and Frontend Architecture Interaction Structure Diagram Appendix}
\end{figure}

\subsection{dApp Contract, Uploader, and Frontend Architecture Structure Diagram}
\begin{figure}[h]
  \vspace{0cm}
  \hspace{-3cm}
  \includegraphics[width=1.4\textwidth]{Secret Raffle Monorepo Structure Diagram v2.png}
  \caption{dApp Contract, Uploader, and Frontend Architecture Structure Diagram.}
  \label{fig:dApp Contract, Uploader, and Frontend Architecture Structure Diagram Appendix}
\end{figure}

\newpage

\subsection{Contract Architecture Structure Diagram}
\begin{figure}[h]
  \vspace{0cm}
  \hspace{-3cm}
  \includegraphics[width=1.4\textwidth]{Secret Raffle Monorepo Contract Structure Diagram.png}
  \caption{Contract Architecture Structure Diagram.}
  \label{fig:Contract Architecture Structure Diagram Appendix}
\end{figure}

\subsection{Uploader Architecture Structure Diagram}
\begin{figure}[h]
  \vspace{0cm}
  \hspace{-3cm}
  \includegraphics[width=1.4\textwidth]{Secret Raffle Monorepo Uploader Structure Diagram.png}
  \caption{Uploader Architecture Structure Diagram.}
  \label{fig:Uploader Architecture Structure Diagram Appendix}
\end{figure}

\newpage

\subsection{Frontend Architecture Structure Diagram}
\begin{figure}[h]
  \vspace{0cm}
  \hspace{-3cm}
  \includegraphics[width=1.4\textwidth]{Secret Raffle Monorepo Frontend Structure Diagram.png}
  \caption{Frontend Architecture Structure Diagram.}
  \label{fig:Frontend Architecture Structure Diagram Appendix}
\end{figure}

\begin{table}[h]
  \centering
  \caption{State Layout}
  \begin{tabular}{@{}lll@{}}
    \toprule
    \textbf{Key} & \textbf{Type} & \textbf{Purpose} \\
    \midrule
    \texttt{ADMIN} & \texttt{Item<CanonicalAddr>} & Contract admin \\
    \texttt{RAFFLE} & \texttt{Item<Raffle>} & Config: end time, price, secret \\
    \texttt{TICKETS} & \texttt{Keymap<CanonicalAddr,u64>} & User ticket balances \\
    \texttt{TOTAL\_TICKETS} & \texttt{Item<u64>} & Total tickets sold \\
    \texttt{WINNER} & \texttt{Item<Option<CanonicalAddr>>} & Selected winner \\
    \texttt{RAFFLE\_STARTED} & \texttt{Item<bool>} & Raffle status \\
    \texttt{WINNER\_SELECTED} & \texttt{Item<bool>} & Winner selection status \\
    \texttt{PRIZE\_CLAIMED} & \texttt{Item<bool>} & Prize claim status \\
    \bottomrule
  \end{tabular}
\end{table}

\end{document}
